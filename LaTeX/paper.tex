% -*- Mode:TeX -*-
% LaTeX template for CinC papers                   v 1.1a 22 August 2010
%
% To use this template successfully, you must have downloaded and unpacked:
%       http://www.cinc.org/authors_kit/papers/latex.tar.gz
% or the same package in zip format:
%       http://www.cinc.org/authors_kit/papers/latex.zip
% See the README included in this package for instructions.
%
% If you have questions, comments or suggestions about this file, please
% send me a note!  George Moody (george@mit.edu)
%
\documentclass[twocolumn]{cinc}
\usepackage{graphicx}
\begin{document}
\bibliographystyle{cinc}

% Keep the title short enough to fit on a single line if possible.
% Don't end it with a full stop (period).  Don't use ALL CAPS.
\title{Leveraging Grover's Algorithm for Efficient Prime Decomposition}

% Both authors and affiliations go in the \author{ ... } block.
% List initials and surnames of authors, no full stops (periods),
%  titles, or degrees.
% Don't use ALL CAPS, and don't use ``and'' before the name of the
%  last author.
% Leave an empty line between authors and affiliations.
% List affiliations, city, [state or province,] country only
%  (no street addresses or postcodes).
% If there are multiple affiliations, use superscript numerals to associate
%  each author with his or her affiliations, as in the example below.

% \author {Jacob Collins$^{1}$, Jaime Raigoza$^{1}$, Sam Siewert$^{1}$ \\
\author {Jacob Collins$^{1}$, Jaime Raigoza$^{1}$, Sam Siewert$^{1}$ \\
\ \\ % leave an empty line between authors and affiliation
 $^1$ California State University, Chico, United States of America (note- full address at end of paper)}

\maketitle

% LaTeX inserts the ``Abstract'' heading in the proper style and
% sets the text of the abstract in italics as required.
\begin{abstract}

  In quantum computing, SP (semiprime) factoring is typically performed 
  with Shor's algorithm, but it is possible to achieve the same results
  more efficientily via Grover's algorithm\cite{whitlock2023quantumfactoringalgorithmusing}. 

  Grover's Algorithm is used to search for one (or many) given state(s)
  among some set of possible outputs. In this case, given some semiprime,
  Grover's algorithm may be used to find the prime factors.

  The abstract with its heading should not be more than 100 mm long,
  which is equivalent to 25 lines of text. Leave 2 line spaces at the
  bottom of the abstract before continuing with the next heading.


\end{abstract}
% LaTeX inserts the extra space here automatically.

\section{Introduction}

What are we doing, broadly?

Include a figure of our circuit diagram and briefly describe what it 
does, and how it does it.

  \subsection{Motivation}

  Why are we replicating the research, why is this important.

  \subsection{Semiprime Factoring}  

  What is SP factoring and why is it important. RSA, etc.

\section{Goals}

Demonstration of quantum advantage, or at least determination of
the problem scale at which quantum advantage will pull ahead. 

\section{Literature Review}

Discuss existing methods of SP factoring- both quantum and classical.

Mention reference paper, QFT arithmetic, CUDA-Q docs, shor's, grover's.

\section{Methodology}
 
Quantum algorithms take advantage of superposition and entanglement,
allowing this form of computation to perform operations on many 
values simultaneously, rather than checking each value individually
as is necessary in classical algorithms.

  \subsection{Using Grover's Algorithm to Invert Functions} 

  Describe how Grover's works and show figures of a generalized
  circuit for Grover's to invert functions

  \subsection{Semiclassical Arithmetic} 

  Describe our original approach- bitwise addition, registers, operations, etc.

  Mention why an alternative approach was needed.

  \subsection{Arithmetic in the Quantum Fourier Domain}  

  Describe the updated methods of quantum arithmetic, and how it solved the 
  previous problems.

  Figures to visualize how the weighted phase shifts work to produce
  expected results matching addition, scaled addition, and register 
  multiplication.

\section{Results and Discussion}

Include some charts showing our runtime and qubit requirements vs N.

  \subsection{Accuracy and Limitations}

  How precise and accurate were our measurements. Mention edge-cases like
  square semiprimes, etc.

  How many qubits were we able to simulate, i.e., how large of semiprimes
  can we factor with our current systems.

  \subsection{Comparison to Shor's Algorithm}

  List strengths and weaknesses of Shor's, i.e., more documentation, more
  examples, larger code base to reference, but ours is faster and uses less
  qubits.

\section{Conclusion}

What have we learned so far, what does it mean, at what problem scale might
we see quantum advantage based on our results, etc.

\section{Future Work}
 
Implementation of the optimization with M, S, p, and q, rather than just 
a and b from N.

\balance

\section*{Acknowledgments}  
% This section is not numbered.
% 
Acknowledge the source paper.


% LateX generates the ``References'' heading automatically and switches
% to 9 point type for the bibliography.  Please  use BibTeX and
% follow the examples in the sample 'refs.bib' file to enter your references.
\bibliography{refs}

% If you don't use BibTeX (why not?) , comment out or remove the previous
% line, and uncomment the following lines up to the ``}\end{bibliography}''
% line below:
%\begin{thebibliography}{99}{ %\small
% \bibitem{tag} (General form) J. K. Author, ``Name of paper,''
%   \emph{Abbrev. Title of
%   Periodical}, vol. x, no. x, pp. xxx--xxx, Abbrev. Month, year. 

% \bibitem{ito}  M. Ito et al., ``Application of amorphous oxide TFT to
%   electrophoretic display,'' \emph{J. Non-Cryst. Solids}, vol. 354, no. 19,
%   pp. 2777--2782, Feb. 2008.
  
% \bibitem{fardel}  R. Fardel, M. Nagel, F. Nuesch, T. Lippert, and
%   A. Wokaun, ``Fabrication of organic light emitting diode pixels by
%   laser-assisted forward transfer,'' \emph{Appl. Phys. Lett.}, vol. 91,
%   no. 6, Aug. 2007, Art. no. 061103.
  
% \bibitem{buncombe} J. U. Buncombe, ``Infrared navigation Part I: Theory,''
%     \emph{IEEE Trans. Aerosp. Electron. Syst.}, vol. AES-4, no. 3,
%     pp. 352--377, Sep. 1944.
      
% Uncomment the following line if you are not using BibTeX.
%}\end{thebibliography}


% LaTeX inserts the ``Address for correspondence'' heading.
\begin{correspondence}
Jacob Collins\\
1565 Filbert Ave\\
jbcollins@csuchico.edu
\end{correspondence}

\end{document}

